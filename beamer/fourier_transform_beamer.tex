\documentclass[aspectratio=169]{beamer}
\usepackage{hyperref}
\usepackage[T1]{fontenc}
\usepackage{NTU}
\usepackage{animate}
\usepackage{xmpmulti}
\usepackage{multimedia}
\usepackage{graphicx}
\usepackage{media9}
\usepackage{listings}


% other packages
\usepackage{latexsym,amsmath,xcolor,multicol,booktabs,calligra}
\usepackage{graphicx,pstricks,listings,stackengine}

\usepackage[backend=biber,style=numeric]{biblatex}
\addbibresource{references.bib}

\author{KOAN, CORDOVAL, BURGAT, GUILLAUMONT }
\title[Présentation du Projet de Développement Logiciel]{Package draw\_circle\_fourier}%Tu es sur du nom ? haha non
\subtitle{Support visuel pour notre présentation}
\institute{Université de Montpellier}
\date{\today}%la présentation est en francais 


% defs
\def\cmd#1{\texttt{\color{red}\footnotesize $\backslash$#1}}
\def\env#1{\texttt{\color{blue}\footnotesize #1}}
\definecolor{deepblue}{rgb}{0,0,0.5}
\definecolor{deepred}{rgb}{0.6,0,0}
\definecolor{deepgreen}{rgb}{0,0.5,0}
\definecolor{halfgray}{gray}{0.55}

\lstset{
    basicstyle=\ttfamily\small,
    keywordstyle=\bfseries\color{deepblue},
    emphstyle=\ttfamily\color{deepred},    % Custom highlighting style
    stringstyle=\color{deepgreen},
    numbers=left,
    numberstyle=\small\color{halfgray},
    rulesepcolor=\color{red!20!green!20!blue!20},
    frame=shadowbox,
}

\usepackage{media9}
\begin{document}

\begin{frame}
    \titlepage
    \vspace{-10pt}
    \begin{figure}[htpb]
        \begin{center}
            \includegraphics[width=0.3\linewidth]{images/UM.png}
        \end{center}
    \end{figure}
\end{frame}


\begin{frame}
\frametitle{BUT }

Le but de ce projet est de créer un module Python capable de dessiner une animation de n'importe quelle image, en utilisant les séries de Fourier. Ceci est inspiré de la vidéo 'Mais qu'est ce qu'une série de Fourier' par 3Blue1Brown sur youtube.
\\


Un intérêt particulier sur les paramètres (degrés d'approximation), la vitesse des vidéos serait d'intérêt. Des tests et des analyses sur diverses images doivent être étudiés. Dans un cas simple, on commencerait par la contrainte que l'image d'entrée peut être un fichier svg noir et blanc. Ensuite, une adaptation aux cas de couleur et au format d'image général (comme png, jpg, etc.) pourrait être ajoutée


\href{https://draw-fourier-circle.readthedocs.io/en/latest/fourier/gallery/examples.html}{\textcolor{blue}{\textbf{DOC}}}


\end{frame}

\begin{frame}
\frametitle{Répartition du travail }

[Kenjy] Traitement des images et conversion en points :

\hspace{1cm} $\bullet \hspace{0.2cm} \text{Extraction des coordonnées à partir d'une image}$

\hspace{1cm} $\bullet \hspace{0.2cm} \text{Coordonnées X et Y -> Nombres Complexes}$

[Chloë] Trouvez les coefficients de Fourier pour localiser approximativement des points donnés:

\hspace{1cm} $\bullet \hspace{0.2cm} \text{Fonction pour générer x + iy à un instant t donné}$

\hspace{1cm} $\bullet \hspace{0.2cm} \text{Fonction pour évaluer y au temps t en utilisant l'approximation de Fourier}$

\hspace{1.4cm} du degré N

\hspace{1cm} $\bullet \hspace{0.2cm} \text{Fonction pour calculer les coefficients}$

\hspace{1cm} $\bullet \hspace{0.2cm} \text{Partie approximation}$

[Pierre + Paul] Créer une animation à partir des coefficients calculés:

\hspace{1cm} $\bullet \hspace{0.2cm} \text{Fonction qui modifie les coefficients afin de tracer des cercles}$

\hspace{1.4cm} dans la dernière fonction.

\hspace{1cm} $\bullet \hspace{0.2cm} \text{Fonction qui affiche le rayon de chaque cercle ainsi que le cercle}$

\end{frame}

  \begin{frame}
\frametitle{Séries et Transformées de Fourier }
\begin{columns}
\begin{column}{0.5\textwidth}

\begin{block}{Equation complexe générique}
Si : $Z(t) = a_{0} + a_{1} e^{it} + a_{2} e^{2it} + \cdots +a_{N} e^{Nit} + a_{-1} e^{-it} + a_{-2} e^{-2it} + \cdots +a_{-N} e^{-Nit}$

avec \begin{itemize}
    \item  $n$ : vitesses des cercles
    \item $a$ : rayons et phases
\end{itemize} 

Alors : $\forall n \in \mathbb{Z}, a_{n} = \frac{1}{\tau} \int_{0}^{\tau} \, Z(t) e^{-int} \, dt$
  
\end{block}

\end{column}
\begin{column}{0.5\textwidth}  %%<--- here
\begin{figure}
    \centering
            \includegraphics[width=0.7\linewidth]{Epicycloid-3.pdf}
            \caption{Epicycloïde}
            \end{figure}

\end{column}
\end{columns}
\end{frame}

\author{Kenjy Koan}
\title[Présentation du Projet de Développement Logiciel]{Manipulation de l'image }

\begin{frame}
    \titlepage
    \vspace{-10pt}
\end{frame}

\begin{frame}
\frametitle{Manipulation de l'image }
\begin{block}{Idée générale}
  \begin{itemize}
      \item Dessin linéaire en 2D $\rightarrow$ fonction paramétrique $f(t) = (\, x(t), y(t) \,)$
      \item Chemin paramétrique passant par les pixels $\rightarrow$ composantes $x$ et $y$
      \item Approximation par la transformée de Fourier \item Reconstitution fidèle de l'image
  \end{itemize}
  \end{block}
\end{frame}

\begin{frame}
\frametitle{Manipulation de l'image }
  \begin{figure}[htpb]
        \begin{center}
            \includegraphics[width=0.45\linewidth]{parametric-plot.png}
            \caption{Illustration de la méthode}
        \end{center}
    \end{figure}
\end{frame}
  
 \begin{frame}
  \frametitle{Manipulation de l'image }
  \begin{figure}[htpb]
        \begin{center}
            \includegraphics[width=0.9\linewidth]{image reader.png}
            \caption{Classe ImageReader}
        \end{center}
    \end{figure}
  
\end{frame}



\author{Chloë Cordoval}

\title[Présentation du Projet de Développement Logiciel]{Coefficients de Fourier}

\begin{frame}
    \titlepage
    \vspace{-10pt}
\end{frame}


\begin{frame}
\frametitle{Coefficients de Fourier}

\begin{block}{Petite introduction }

Dans cette partie, nous trouverons les coefficients de Fourier pour localiser approximativement des points donnés.

Pour cela, nous avons créé une classe de fonctions nommée \textcolor{red}{\textbf{Fourier}} pour notre Package. 


\end{block}

\begin{block}{Séries de Fourier} 

Pour une période L, les séries de Fourier complexes sont : $$\boxed{f(t) = \sum_{i=-\infty}^{\infty} C_{n} e^{i n t \frac{2\pi}{L}}}$$.

\end{block}
\end{frame}

\begin{frame}{Séries de Fourier}
    
\begin{block}{Séries de Fourier} 
On retrouve cette fonction à partir \textcolor{red}{\textbf{ des coefficients de Fourier}} :

$$\boxed{Cn = \frac{1}{L}\int^{-\pi}_{\pi}e^{- i n t\frac{2\pi}{L}}f(t)dt}$$

\end{block}
\end{frame}

 \begin{frame}
  \frametitle{Manipulation de l'image }
  \begin{figure}[htpb]
        \begin{center}
            \includegraphics[width=0.7\linewidth]{2.png}
            \caption{Classe Fourier}
        \end{center}
    \end{figure}
  
\end{frame}



\begin{frame}{Description de la classe}

Dans cette classe, nous retrouvons deux fonctions qui sont :
\vskip 0.3cm
\hspace{5cm} $\bullet \hspace{0.2cm}$ \textbf{Coeff\_List}
\vskip 0.3cm
\hspace{5cm} $\bullet \hspace{0.2cm}$ \textbf{DFT}
\vskip 0.3cm
La fonction \textcolor{red}{\textbf{Coeff\_List}} prend comme paramètres d'entrée, les valeurs \textbf{time\_table},  \textbf{x\_table}, \textbf{y\_table}, \textbf{order}.
\vskip 0.3cm
Ces valeurs sont les paramètres de sortie de la fonction \textcolor{blue}{\textbf{get\_tour}} de la classe  \textcolor{red}{\textbf{ImageReader}} décrite précédemment.
\vskip 0.3cm
Cette fonction a pour but de calculer les coefficients de Fourier complexes et les décomposer en une partie réelle et imaginaire. 

\end{frame}


\begin{frame}{Description de la classe}

C'est pour cela que nous avons introduite hors du package une fonction \textcolor{red}{\textbf{f}} qui va convertir les coordonnées trouvées précédemment en nombres complexes par rapport à un temps t donné.
\vskip 0.3cm
Nous allons le retrouver dans notre boucle "for" dans la fonction \textbf{Coef\_List} car c'est elle qui va être décomposée en partie réelle et imaginaire. 
\vskip 0.3cm
Notre fonction retournera les valeurs de \textbf{coef\_list} sous forme d'un array qui a été initialisé avant notre itération comme une liste vide.
\vskip 0.3cm
Nous aurons donc les coefficients de Fourier complexes voulus sous forme d'un tableau (array) de valeurs élémentaires. 
\end{frame}

\begin{frame}{Transformée de Fourier}
    
\begin{block}{Séries de Fourier} 
Soit une fonction f de $\mathrm{L}^{1}(\mathbb{R})$. On appelle  \textcolor{red}{\textbf{transformée de Fourier}} de f, notée F(t), la fonction : 

$$\boxed{F(t) = \frac{1}{\sqrt{2\pi}}.\int^{\infty}_{-\infty} e^{-ixt}f(x)dx}$$

\end{block}
\end{frame}

\begin{frame}{Description de la classe}

La fonction \textcolor{red}{\textbf{DFT}} pour "Discrete Fourier Transform", prend comme paramètres d'entrée, les valeurs \textbf{coef\_list},  \textbf{t}, \textbf{order}.


\vskip 0.3cm
\textbf{coef\_list} est le paramètre de sortie de la fonction \textcolor{red}{\textbf{Coef\_List}} définie dans notre classe. 
\vskip 0.3cm

Cette fonction va simplement calculer les transformées de Fourier de nos coefficients trouvés précédemment. 

\vskip 0.3cm
Elle va retourner comme paramètres de sortie, la partie réelle et la partie imaginaire de la transformée.  

\end{frame}



\author{Burgat Paul et Guillaumont Pierre}

\title[Présentation du Projet de Développement Logiciel]{Visualisation des cercles de rotation de Fourier}

\begin{frame}
    \titlepage
    \vspace{-10pt}
\end{frame}

\begin{frame}
\frametitle{Visualisation des cercles de rotation de Fourier}

\begin{block}{Introduction}
  Dans cette partie, à partir des coefficients de Fourier trouvés pour localiser approximativement des points donnés, nous allons faire une animation qui va tracer une courbe représentant une image choisie.
  Pour se faire, nous avons créé une classe de fonctions \textcolor{red}{\textbf{DrawAnimation}}.
\end{block}

\end{frame}
\begin{frame}{Description de la classe}
\vskip 0.3cm

La classe \textcolor{red}{\textbf{DrawAnimation}} est composée de 4 fonctions :
\vskip 0.3cm
$\bullet \hspace{0.2cm}$ \textbf{visualize}
\vskip 0.3cm
$\bullet \hspace{0.2cm}$ \textbf{update\_c}
\vskip 0.3cm

$\bullet \hspace{0.2cm}$ \textbf{sort\_velocity}
\vskip 0.3cm

$\bullet \hspace{0.2cm}$ \textbf{animate}
\vskip 0.3cm

\end{frame}

\begin{frame}

  \frametitle{Manipulation de l'image }
  \begin{figure}[htpb]
        \begin{center}
            \includegraphics[width=0.6\linewidth]{5.png}
            \caption{Class\_DrawAnimation}
        \end{center}
    \end{figure}


\end{frame}

\begin{frame}{Fonction visualize}
    

La fonction \textbf{visualize} prend comme variables x\_DFT et y\_DFT qui proviennent de la fonction \textbf{DFT} de la classe \textcolor{red}{\textbf{Fourier}},  ensuite elle prend comme paramètres \textbf{coef} et \textbf{order}.

\textbf{coef} représente les coefficients de Fourier et \textbf{order} nous donnera le nombre de coefficients que l'on souhaite, si \textbf{order}=50, nous aurons 100 coefficients de Fourier.
La variable \textbf{space} représente une liste de données allant de 0 à \textbf{tau}, \textbf{tau} est une constante de cercle égale à 2$\pi$, le rapport de la circonférence d un cercle à son rayon, et pour finir \textbf{fig\_lim} représente les limites des coordonnées x et y de notre figure. 
\newline
Cette fonction a pour but de faire une visualisation qui représente une image choisie à l'aide des cercles de rotation de Fourier. 
\end{frame}
\begin{frame}{Fonction update\_c}
Pour réaliser cela, nous définissons la fonction \textbf{update\_c} qui prend comme variables \textbf{c} et \textbf{t}. La variable \textbf{c} est un ndarray qui nous le verrons plus tard correspond au paramètre \textbf{coef} de la fonction \textbf{visualize} et la variable \textbf{t} est un entier que nous définierons plus tard. Ensuite on itinialise un nouveau ndarray \textbf{new\_c} dans cette fonction, le but de cette fonction est d'arranger les parties réelles et imaginaires des coefficients de Fourier afin de pouvoir afficher le rayon de chaque cercle ainsi que le cercle dans la fonction \textbf{animate}.

\end{frame}

\begin{frame}
    
    \frametitle{Manipulation de l'image }
  \begin{figure}[htpb]
        \begin{center}
            \includegraphics[width=0.6\linewidth]{6.png}
            \caption{Class\_DrawAnimation}
        \end{center}
    \end{figure}
    
\end{frame}

\begin{frame}{Fonction sort\_velocity}
La fonction \textbf{sort\_velocity} prend en entrée la variable \textbf{order} définie dans la fonction \textbf{visualize}. Elle permet de créer une variable \textbf{idx} qui est utilisée dans la fonction \textbf{animate}. On initialise \textbf{idx} comme une liste vide puis on la remplit grâce à une boucle for pour i=1 de \textbf{order}+1 à \textbf{order}-1 jusqu'à pour i=\textbf{order}+1 de $2\times \textbf{order}$ à 0.
    
\end{frame}

\begin{frame}{Fonction animate}

La fonction \textbf{animate} prend en entrée un entier \textit{i}. Elle permet d'afficher le rayon de chaque cercle ainsi que le cercle grâce à une double boucle  \textbf{for} imbriqué. A chaque itération, on affiche un cercle supplémentaire et son rayon. \\
Duquel on affichera le prochain cercle via les variables \textbf{pos} et \textbf{new\_pos} 
Cette fonction trace aussi l'approximation de l'image en utilisant la commande \textbf{line.set\_data}.\\
\bigskip
Pour avoir un rendu animé, on utilise la fonction \textbf{FuncAnimation} du package \textbf{matplotlib.animate}.

\end{frame}


\author{Koan Kenjy, Cordoval Chloë, Burgat Paul, Guillaumont Pierre}

\title[Présentation du Projet de Développement Logiciel]{Exemples}

\begin{frame}
    \titlepage
    \vspace{-10pt}
\end{frame}



\begin{frame}{EXEMPLES}

\href{https://github.com/Chloe971/EXEMPLE/blob/main/exemple1.ipynb}{\textcolor{blue}{\textbf{Coeur-1}}}
\href{https://github.com/Chloe971/EXEMPLE/blob/main/gif/Heart.gif}{\textcolor{blue}{\textbf{Coeur-2}}}
\vskip 0.1cm
\href{https://github.com/Chloe971/EXEMPLE/blob/main/exemple2.ipynb}{\textcolor{blue}{\textbf{Star-1}}}
\href{https://github.com/Chloe971/EXEMPLE/blob/main/gif/star.gif}{\textcolor{blue}{\textbf{Star-2}}}
\vskip 0.1cm
\href{https://github.com/Chloe971/EXEMPLE/blob/main/exemple3.ipynb}{\textcolor{magenta}{\textbf{Horse-1}}}
\href{https://github.com/Chloe971/EXEMPLE/blob/main/gif/Horse.gif}{\textcolor{magenta}{\textbf{Horse-2}}}
\vskip 0.1cm
\href{https://github.com/Chloe971/EXEMPLE/blob/main/exemple4.ipynb}{\textcolor{red}{\textbf{Velo-1}}}
\href{https://github.com/Chloe971/EXEMPLE/blob/main/gif/velo.gif}{\textcolor{red}{\textbf{Velo-2}}}
\vskip 0.1cm
\href{https://github.com/Chloe971/EXEMPLE/blob/main/exemple5.ipynb}{\textcolor{blue}{\textbf{Fish-1}}}
\href{https://github.com/Chloe971/EXEMPLE/blob/main/gif/Fish.gif}{\textcolor{blue}{\textbf{Fish-2}}}
\vskip 0.1cm
\href{https://github.com/Chloe971/EXEMPLE/blob/main/exemple6.ipynb}{\textcolor{red}{\textbf{Bart-1}}}
\href{https://github.com/Chloe971/EXEMPLE/blob/main/gif/Bart.gif}{\textcolor{red}{\textbf{Bart-2}}}
\vskip 0.1cm
\href{https://github.com/Chloe971/EXEMPLE/blob/main/exemple7.ipynb}{\textcolor{magenta}{\textbf{Papillon-1}}}
\href{https://github.com/Chloe971/EXEMPLE/blob/main/gif/Buuterfly.gif}{\textcolor{magenta}{\textbf{Papillon-2}}}
\vskip 0.1cm
\href{https://github.com/Chloe971/EXEMPLE/blob/main/exemple8.ipynb}{\textcolor{magenta}{\textbf{Akatsuki-1}}}
\href{https://github.com/Chloe971/EXEMPLE/blob/main/gif/Akatsuki.gif}{\textcolor{magenta}{\textbf{Akatsuki-2}}}
\vskip 0.1cm
\href{https://github.com/Chloe971/EXEMPLE/blob/main/exemple9.ipynb}{\textcolor{blue}{\textbf{Apple-1}}}
\href{https://github.com/Chloe971/EXEMPLE/blob/main/gif/Apple.gif}{\textcolor{blue}{\textbf{Apple-2}}}
\vskip 0.1cm
\href{https://github.com/Chloe971/EXEMPLE/blob/main/exemple10.ipynb}{\textcolor{red}{\textbf{Rolex-1}}}
\href{https://github.com/Chloe971/EXEMPLE/blob/main/gif/Rolex.gif}{\textcolor{red}{\textbf{Rolex-2}}}
\vskip 0.1cm
\href{https://github.com/Chloe971/EXEMPLE/blob/main/exemple11.ipynb}{\textcolor{magenta}{\textbf{Puma-1}}}
\href{https://github.com/Chloe971/EXEMPLE/blob/main/gif/Puma.gif}{\textcolor{magenta}{\textbf{Puma-2}}}








    
\end{frame}































\end{document}
